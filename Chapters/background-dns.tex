\section{DNS}
\label{sec:background-dns}

In order to resolve a human-friendly URL to a set of actual server IP addresses, a Domain Name System (DNS) Server query is required. 
In such a query the client asks its local DNS server to return known IPs for a given URL. 
In case a list of multiple IP addresses is returned, the client usually uses only the first one and discards all other IPs. 
Note, that different DNS servers might return different sets of IPs, \eg results from an ISP's local DNS server might differ from Google's DNS server. 

According to the previous \mhttp~study~\cite{KIM13-MHTTP}, even with a single query to the local ISP's DNS Server, for the top $1000$ websites provided by \term{Alexa.com}~\cite{URL-ALEXA}, approximately $30$\perc are associated to more than one IP address. 
Further, $10$\perc and $5$\perc belong to different network prefixes and different Autonomous Systems (ASes), respectively. 

The study also shows that performing two lookups, one to the local DNS and one to Google's DNS, increases these fractions to $35$\perc (IP addresses), $17$\perc (prefixes) and $7$\perc (ASes). 

Taking advantage of these results is done by \mhttp's \term{MultiDNS} module and is briefly explained in~\chref{ch:vanilla}. 
