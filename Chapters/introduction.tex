\chapter{Introduction} % Main chapter title
\label{ch:introduction}

\lhead{Chapter I. \emph{Introduction}} % Change X to a consecutive number; this is for the header on each page - perhaps a shortened title

%- multiple interfaces / sources, but HTTP does not take advantage of that

%- take advantage of network diversity

%- mHTTP Vanilla already implemented, but not complete for proper evaluations and traffic scheduling


Multi-source Multipath HTTP (\mhttp)~\cite{JKIM14-TUND}\cite{KIMSIG} aims to download a single content from different sources through multiple client interfaces at the same time, intending to reduce the overall download times of web contents for end-users. 
Two major recent developments make such an approach feasible. 
First, the majority of mobile devices has two interfaces (\ie \wifi~and \lte) and 
second, popular content tends to be distributed in Content Distribution Networks (CDN), thus multiple copies of the same popular content exist in several locations. 

Previous studies on \mhttp~\cite{JKIM14-TUND}\cite{KIMSIG} were conducted on single file downloads in controlled testbeds. 
Website downloads and real-world measurements pose a very interesting benchmark to verify, whether \mhttp~indeed has the potential to increase the end-user's every day browsing experience, however, these evaluations are uncharted in the previous studies~\cite{JKIM14-TUND}\cite{KIMSIG}. 
Further, no intelligent chunk scheduling over the available paths was evident, thus different link performances were not considered to leverage the full potential of \mhttp. 

Continuing the previous works~\cite{JKIM14-TUND}\cite{KIMSIG}, we propose a new \mhttp~prototype (\protonew), which is based on the original prototype (\ie \protoold). 
\protonew~comes with the same benefits as its predecessor \protoold, \eg

\begin{itemize}
\item only client socket API modifications are necessary
\item taking advantage of all kinds of network diversities
\item full transparency towards the application layer, \ie applications do not need any special modifications in order to work with \mhttp.
\end{itemize}

In addition to that \protonew~provides 

\begin{itemize}
\item algorithms for intelligent chunk scheduling over the available paths. 
\item an advanced memory management and implementation, allowing to deliver data to the application in less system calls than \protoold.
\item the ability to download websites from servers. 
\item an advanced threading model, enabling quick TCP buffer draining to prevent processing bottlenecks having an effect on the used bandwidth. 
\end{itemize}

In the previous studies~\cite{JKIM14-TUND}\cite{KIMSIG}, each download scenario was repeated several times for different fixed chunk sizes over the paths in order to determine the best performing chunk size. 
Obviously, it is desirable to have an algorithm that automatically figures out the best performing chunk size during download time, without repeating the download and without any prior knowledge about the paths. 
We show that \protonew's intelligent chunk scheduling performs in any download scenario at least as good as the best performing fixed chunk size of \protoold, thus enabling \mhttp~to automatically determine the best traffic distribution during download time without prior knowledge of the paths. 

Moreover, we compare \protonew~to MPTCP and show that it performs similar, thus being a viable alternative. 

Finally, we evaluate \protonew~on popular public websites and show that \mhttp~indeed has the potential of reducing download times for websites with few relatively large embedded objects, while not doing any harm to websites with very small objects only. 

