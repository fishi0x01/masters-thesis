\chapter{Related Work}
\label{ch:related-work}

\lhead{Chapter III. \emph{Related Work}}

\mhttp~is not the first approach that tries to take advantage of network diversity and deals with chunk scheduling. 
In this chapter we discuss technologies that also try to enhance download speed by using multiple paths, focusing on their chunk scheduling approaches and how we can learn from them.

\strong{Early studies}
One of the early studies by Rodriguez and Biersack~\cite{RODRIGUEZ02-DPA} studies the concept of parallel access to replicated content. 
In their study they use a chunk scheduler that is very similar to \protoold's \algbase~scheduler, which will be further introduced in~\xref{sec:baseline-scheduler}.
Moreover, they mention the problem of idle connections towards the end of a download, thus inspiring us for a connection synchronization approach as introduced in~\xref{sec:synchronization-approach}.

\strong{SCTP}
The Stream Control Transmission Protocol (SCTP)~\cite{IYENGAR06-CMT} enables concurrent multipath transfer. 
Depending on the platform, usually simple scheduler approaches such as Round-Robin or First-Come, First-Serve are used~\cite{RFC-4960}. 
Another study discusses the importance of a pluggable scheduler for easy customizations based on the application need~\cite{YWANG11-INFO}.
This shows that a clean and easy scheduler API can simplify the process of designing and implementing different new scheduling algorithms. 

\strong{MPTCP}
Multipath TCP (MPTCP~\cite{RFC-MPTCP,RAICIU12-HHC}) is a multipath approach between a single server and a single client, which operates on the TCP layer. 
MPTCP Head-Of-Line Blocking, \ie a line of packets blocked by the first one, due to periodic outages on a bad path, is one major concern for optimal scheduling. 
Further, MPTCP might suffer from receive-window limitations, \ie not enough memory to accommodate out-of-order packets~\cite{PAASCH14-MPTCP}. 
Studies have shown, that scheduler algorithms such as Lowest-RTT-First, which take the link quality in terms of round-trip-latency into account, tend to perform better in MPTCP than simple algorithms like Round-Robin~\cite{PAASCH14-MPTCP}.
We discuss important link quality metrics for \mhttp~and how we intent to measure them in~\xref{sec:metrics}. 

\strong{Video streaming}
Studies in video streaming for different providers (\eg Youtube) were performed, in which chunk scheduling algorithms were proposed. 
Evensen~\etal evaluate fixed chunk and dynamic chunk scheduling~\cite{EVENSEN11-ITP}. 
Their video streaming client benefits from dynamic chunk scheduling, because less traffic is distributed on the slow path. 
Chen~\etal propose a dynamic chunk scheduler that doubles or halves the chunk size, based on the evolution of the link quality, \ie the bandwidth~\cite{CHEN14-MMA}. 
Inspired by this, we propose a similar approach that also takes the round-trip-latency and bandwidth into account in~\xref{sec:alpha-approach}. 

%\strong{Download boosters}
%\todo{torrent, JDownloader, samsung galaxy booster}
%\todo{scheduler unknown, but likely simple and very greedy - dishonors the fairness aspect of TCP}

\strong{Socket Intents}
Socket Intents uses pre-configurable policies to decide which interface to use for which kind of traffic~\cite{SCHMIDT13-SIL}. 
It distinguishes between different types of traffic, \eg bulk transfers (big data) and query transfers (small data). 
Depending on the traffic type a primary interface is selected, \eg query transfers preferably over low latency interfaces and bulk transfers preferably over high throughput interfaces. 
In order to accomplish this, the application layer needs to provide additional information about the traffic. 
